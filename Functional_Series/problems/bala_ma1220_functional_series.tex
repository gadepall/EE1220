\documentclass[journal,12pt,twocolumn]{IEEEtran}
\usepackage{amsmath}
\usepackage{amssymb}
\usepackage{enumerate}
\usepackage[utf8]{inputenc}
\usepackage{multicol}
\usepackage{iithtlc}
\begin{document}
\providecommand{\nCr}[2]{\,^{#1}C_{#2}} % nCr
\providecommand{\nPr}[2]{\,^{#1}P_{#2}} % nPr
\providecommand{\mbf}{\mathbf}
\providecommand{\pr}[1]{\ensuremath{\Pr\left(#1\right)}}
\providecommand{\qfunc}[1]{\ensuremath{Q\left(#1\right)}}
\providecommand{\sbrak}[1]{\ensuremath{{}\left[#1\right]}}
\providecommand{\lsbrak}[1]{\ensuremath{{}\left[#1\right.}}
\providecommand{\rsbrak}[1]{\ensuremath{{}\left.#1\right]}}
\providecommand{\brak}[1]{\ensuremath{\left(#1\right)}}
\providecommand{\lbrak}[1]{\ensuremath{\left(#1\right.}}
\providecommand{\rbrak}[1]{\ensuremath{\left.#1\right)}}
\providecommand{\cbrak}[1]{\ensuremath{\left\{#1\right\}}}
\providecommand{\lcbrak}[1]{\ensuremath{\left\{#1\right.}}
\providecommand{\rcbrak}[1]{\ensuremath{\left.#1\right\}}}
\newcommand{\sgn}{\mathop{\mathrm{sgn}}}
\providecommand{\abs}[1]{\left\vert#1\right\vert}
\providecommand{\res}[1]{\Res\displaylimits_{#1}} 
\providecommand{\norm}[1]{\lVert#1\rVert}
\providecommand{\mtx}[1]{\mathbf{#1}}
\providecommand{\mean}[1]{E\left[ #1 \right]}
\providecommand{\fourier}{\overset{\mathcal{F}}{ \rightleftharpoons}}
%\providecommand{\hilbert}{\overset{\mathcal{H}}{ \rightleftharpoons}}
\providecommand{\system}{\overset{\mathcal{H}}{ \longleftrightarrow}}

\newcommand{\solution}{\noindent \textbf{Solution: }}
\providecommand{\dec}[2]{\ensuremath{\overset{#1}{\underset{#2}{\gtrless}}}}
\title{ 
\logo{
Problem Set: Functional Series
}
}
\author{J.~Balasubramaniam$^{\dagger}$ %<-this  stops a space
\thanks{$\dagger$ The author is with the Department of Mathematics, IIT Hyderabad
502285 India e-mail: jbala@iith.ac.in. }
}
\maketitle

\begin{enumerate}
\item Discuss the convergence of the series $\sum\limits_{n=1}^{\infty} \frac{(-1)^n x^{2n}}{4^n n^p}$ w.r.to both $x$ and an arbitrary constant $p$. 

\item Expand in Fourier series the function $f(x) =\frac{\pi^2}{12}-\frac{x^2}{4}$ in the interval $[-\pi,\pi]$ and hence show that
$\frac{\pi^2}{12}=\sum\limits_{n=1}^{\infty} (-1)^{(n+1)} \frac{1}{n^2}$.

\item Show using Taylor’s series that $e^{i\theta} = \sin\theta + i \ \cos\theta$. 


\item Give examples of the following, with a brief justification. 
\setlength\itemsep{2em}
\begin{enumerate}[(a)]
\item A function $f(x)$ and a point $x_0$ such that $\frac{df}{dx}(x_0)=0$ but $x_0$ is not an extreme point.

\item A function $f(x)$ and a point $x_0$ such that $x_0$ is an extreme point but $\frac{df}{dx}(x_0)$ does not exist.


\item A function $f(x)$ and a point $x_0$ such that $\frac{df}{dx}$ is continuous at $x_0$ but not differentiable at $x_0$.

\item A power series such that it is convergent only at $x = 1$.
\item A function $f(x)$ that is nowhere piecewise monotonic.
\item A sequence of functions $\lbrace f_n \rbrace$ defined everywhere (i.e., for all $x \in \mathbb{R}$) but whose functional series $ \sum_{n=1}^{\infty} f(x)$ does \textbf{not} converge anywhere (i.e., for any $x \in \mathbb{R}$).
\item A sequence of functions $\lbrace f_n \rbrace$ to show that the converse of the following statement is \textbf{not true}:
\textit{”If a sequence of continuous functions $\lbrace f_n \rbrace$ converges uniformly to $f$, then $f$ is continuous.”}

\end{enumerate}

\item Determine the exact intervals of convergence for the following:


\begin{enumerate}[(i)]
\begin{multicols}{2}
\setlength\itemsep{2em}

\item $
\sum\nolimits n^2x^2
$
\item $
\sum\nolimits {\dfrac{2^n}{x^2}}x^n
$
\item $
\sum\nolimits \dfrac{x^n}{n^n}
$
\item $
\sum\nolimits {\dfrac{1}{(n+1)^2 2^n}}x^n
$
\item $
\sum\nolimits {\dfrac{(-1)^n}{n^24^n}}x^n
$
\item $
\sum\nolimits \sqrt{nx^n}
$
\item $
\sum\nolimits {\dfrac{3^n}{n4^n}}x^n
$
\item $
\sum\nolimits {\dfrac{n^3}{3^n}}x^n
$
\item $
\sum\nolimits {\dfrac{3^n}{\sqrt{n}}}x^{2n+1}
$
\item $
\sum\nolimits x^{n\,!}
$
\end{multicols}
\end{enumerate}

\item Consider a power series $\sum\nolimits {a_n}x^n$ with radius of convergence $\mathbb{R}$.

\setlength\itemsep{2em}
\begin{enumerate}[(a)]

\item Prove that if all the coefficients $a_{n}$ are integers and if infinitely many of them are non-zeros, then $\mathbb{R}\leqslant1 $.	

\item If $|{a_n}|^ {\frac{1}{n}} \rightarrow l$,then 
$
\mathbb{R} = 
\begin{cases} 0 &   l = \infty 
\\ 
\infty&    l = 0 
\\ 
\frac{1}{l}&  0 < l < \infty
\end{cases} 
$

\item If ${a_n}\neq0$ for all large $n$ and $ {\dfrac{|{a_{n+1}}|}{|{a_n}|}} \rightarrow l$,the conclusion of (b) above still holds.

\item Verify the above with the following series whose co-efficients are given as:

\begin{enumerate}[(i)]

\item $a_{n} = \dfrac{n^3}{3^n} $
\item $a_{n} = \dfrac{2^n}{n\,!}$
\item $a_{2n-1} = \dfrac{1}{4^n}; a_{2n} = \dfrac{1}{9^n}$


\end{enumerate}

\end{enumerate}

\item Consider a power series $\sum\nolimits a_{n}x^n$ with a finite radius of convergence $\mathbb{R}$. Prove that if all the coefficients $a_{n}\geqslant0$ for all \textit{n} and if the series converges at $\mathbb{R}$, then the series also converges at $\mathbf{-R}$.

\item For each $\textit{n}\in\mathbb{N}$, let $f_{n} = (\cos x)^n$. Show that

\begin{enumerate}[(a)]

\item each $f_{n}$ is continuous.
\item $\lim f_{n}(x) = 0$ unless \textit{x} is a multiple of $\pi$.
\item $\lim f_{n}(x) = 1$ if \textit{x} is an even multiple of $\pi$.
\item $\lim f_{n}(x)$ doesnot exist if \textit{x} is an odd multiple of $\pi$.

\end{enumerate}

\item For each $\textit{n}\in\mathbb{N}$, let $f_{n} = \frac{1}{n}\sin x$. Show that

\begin{enumerate}[(a)]

\item each $f_n$ is differentiable.
\item $\lim f_n(x) = 0$ for all $\textit{x} \in \mathbb{R}$.
\item $\lim f'_{n}(x)$ need not exist (for instance at $x = \pi$).

\end{enumerate}

\item For each $\textit{n} \in \mathbb{N}$, let $f_n(x) = nx^n$ for $\textit{x} \in [0,1]$. Show that

\begin{enumerate}[(a)]

\item $\lim f_n(x) = 0$ for all $\textit{x} \in [0,1]$.
\item $\lim\limits_{n\to\infty}\int\limits_{0}^{1} f_n(x)dx = 1$.

\end{enumerate}

\item For each $\textit{n} \in \mathbb{N}$, let $f_n(x) = {\bigg(x-{\dfrac{1}{n}}\bigg)^2}$ for $\textit{x} \in [0,1]$.

\begin{enumerate}[(a)]

\item Find $f(x) = \lim f_n(x)$.
\item Does $(f_n)$ converge pointwise on $[0,1]$?
\item Does it also converge uniformly?
\end{enumerate}

\item Obtain the Taylor series of the following functions about the indicated point a:

\begin{enumerate}[(i)]
\begin{multicols}{2}
\setlength\itemsep{2em}

\item $
\tan x;  
a = \frac{\pi}{4}
$
\item $
{e}^{\sin x};
 a = 0
 $
\item $
\ln(\cos x) 
a = 0
$
\item $
\cos ^2x;  
a = 0
$
\item $
\cos ^2x;
a = \frac{\pi}{4}
$
\item $
\dfrac{1}{x^3}; 
a = 7
$
\item $
\dfrac{a}{1+{x^4}}; 
a = 3
$
\item $
\tan ^{-1} x; 
a = 0
$

\end{multicols}
\end{enumerate}

\item Suppose that $f_n$ is differentiable on an interval $\textit{I}$ centered at $x = a$ and that \\ 
{\begin{equation*} g(x) = b_0 + b_1(x-a) + ....+ b_n(x-a)^n, \end{equation*}}  
is a polynomial of degree \textit{n} with constant coefficients $b_0,b_1,....,b_n$.Let $E(x) = f(x)-g(x)$. Show that if

\begin{enumerate}[(a)]
\item $E(a)=0$ (i.e., the approximation error is zero at $x = a$).

\item $\lim\limits_{x \to a}{\dfrac{E(x)}{(x-a)^n}} = 0$ (i.e.,the error is negligible when compared to $(x-a)^n$) \\ then $b_k = \dfrac{f'(a)}{k\,!}, k = 0,....,n$. Thus the Taylor's polynomial is the only polynomial of degree less than or equal to \textit{n} whose error is zero at $x = a$ and negligible when compared to $(x-a)^n$.

\end{enumerate}

\item Find the Fourier Series of the following functions:
%\setlength{\columnsep}{3cm}
\begin{enumerate}[(i)]
\setlength\itemsep{2em}
\begin{multicols}{2}
{\scriptsize

\item $
f_x = x^3;
{{-\pi} \leqslant x \leqslant \pi}
$
\item $
f_x = x+{|x|}; x \in \sbrak{-\pi,\pi}
% {{-\pi} \leqslant x \leqslant \pi}
 $
\item $
f_x =
\begin{cases} 1;
 {-\frac{\pi}{2}} \leqslant x \leqslant {\frac{\pi}{2}}
  \\
  -1;
   {\frac{\pi}{2}} < x \leqslant \frac{3\pi}{2} 
  \end{cases}
  $
\item $
f_x = 
\begin{cases} 1;
 {-1} \leqslant x \leqslant 0 
 \\ {-1}; 
  0 < x \leqslant 1 
  \end{cases}
  $
\item $
f_x =
\begin{cases} x; 
 {-2}\leqslant x < 0 
 \\
  {{\pi}-x}; 
  0 < x \leqslant 2 
  \end{cases}
  $
\item $
f_x = {x^3} ;
{-2} \leqslant x \leqslant 2 
$
%\item 
%$
%\!
%\begin{aligned}[t]
%f_x &= x + |x|; 
%\\
%&{\frac{\pi}{2}} \leqslant x \leqslant {\frac{\pi}{2}}
%\end{aligned}
%
\item $
f_x = x + |x|; 
{\frac{\pi}{2}} \leqslant x \leqslant {\frac{\pi}{2}}
$
\item $
f_x = 
\begin{cases} x; 
{-\frac{\pi}{2}} \leqslant x \leqslant {\frac{\pi}{2}} 
\\
{{\pi} - x};
x \in \sbrak{-\frac{\pi}{2},\frac{3\pi}{2}}
%{-\frac{\pi}{2}} < x \leqslant {\frac{3\pi}{2}} 
\end{cases} 
$ 

}
\end{multicols}
\end{enumerate}

\item Expand the following functions such that we obtain (i) only a sine series and (ii) only a cosine series:

\begin{enumerate}[(i)]
\setlength\itemsep{2em}
\begin{multicols}{2}
{\scriptsize
\item $
f_x = x;
 0 \leqslant x \leqslant {2\pi}
 $
\item $
f_x = {\pi - x}; 
0 \leqslant x \leqslant \pi
$
\item $
f_x = \sin^2x ;
0 \leqslant x \leqslant \pi
$
\item $
f_x = e^x;
 0 \leqslant x \leqslant L
 $
\item $
f_x = x^2;
 0 \leqslant x \leqslant L
 $
\item $
f_x = {4-x^2};
 0 \leqslant x \leqslant L
 $
}
\end{multicols}
\end{enumerate}

\item Find the Fourier Series of $f_x = {\dfrac{(\pi - x)^2}{4}}$ on $0 \leqslant x \leqslant 2\pi$ and hence show that

\begin{enumerate}[(a)]
\setlength\itemsep{2em}
\begin{multicols}{2}
{\scriptsize
\item $
\dfrac{{\pi}^2}{6} = {\dfrac{1}{1^2}} + {\dfrac{1}{2^2}} + {\dfrac{1}{3^2}} +...
$
\item $
\dfrac{\pi^2}{12} = {\dfrac{1}{1^2}} - {\dfrac{1}{3^2}} + {\dfrac{1}{5^2}} -...
$
\item $
\dfrac{\pi^2}{8} = {\dfrac{1}{1^2}} + {\dfrac{1}{3^2}} + {\dfrac{1}{5^2}} +...
$
}
\end{multicols}
\end{enumerate} 

\item Find the Fourier Series of 
$f_x = \sqrt{1-\cos x}$ on $(0,2\pi)$ 
and hence deduce that 
\begin{equation*}
\dfrac{1}{2} = \sum_{n=1}^{\infty} {\dfrac{1}{4n^2-1}}
\end{equation*}


\item Let $f$ be a periodic function with period $2\pi$. Let $f_n$ be the trignometric polynomial of order $n$ given as follows \\
{\begin{equation*} 
f_n(x) = a_0 + \sum_{k=1}^{n} a_k \cos kx + b_k\sin kx.
\end{equation*}}


\item Show that if $f_n$ minimizes the integral of the square of the error in approximating \textit{f},viz.,\\
{\begin{equation*}
\int_{-\pi}^{\pi}{[f(x)-f_n(x)]^2}dx, 
\end{equation*}}
then the coefficients of $f_n$ are given as Fourier coefficients.

\item Say True or False, with a brief justification. \\


\begin{enumerate}[(a)]
\setlength\itemsep{2em}

\item The series $\sum\limits_{n=1}^{\infty}\frac{x^n}{n}$ is dominated on $0 \leqslant x \leqslant 1$.
\item A function $f$ is Riemann integral over $[a,b]$ if and only if $f$ is continuous over $[a,b]$ .
\item $\int_{0}^{1}x^m(1-x)^n \ dx = \int_{0}^{1}x^n(1-x)^m \ dx$ for any $m>0, n>0$.
\item $e^{i\theta} = \sin\theta + i \ \cos\theta$.
\item $\int_{-\infty}^{+\infty} f(x) \ dx = \lim\limits_{b \to \infty}\int_{-b}^{+b}f(x) \ dx$ for any real function $f$.\\
\end{enumerate}

\item Give example of  \\

\begin{enumerate}[(a)]
\setlength\itemsep{2em}

\item a power series whose radius of convergence is $(-3,3)$.
\item a functional series that is pointwise convergent but not uniformly convergent.
\item a function $f$ and an interval $[a,b]$ such that $f$ is not Riemann integral over $[a,b]$.
\item an infinitely differentiable function whose Taylor's series does not converge to it.
\item two functions $ \phi(x)$ and $\psi(x)$ and an interval $[a,b]$ such that $\int_{a}^{b} \phi(x) \cdot \psi(x) dx=0$.\\
\end{enumerate}


\item Find the radius of convergence of the power series $\sum\limits_{n=2}^{\infty}\frac{x^n}{n\ln n}$. \\


\item Discuss the Maclaurin's series expansion of $f(x)=(1+x)^m$ for any $m \in \mathbb{R}$ and hence find a series expansion of $\sin ^{-1} x$. \\

\item Expand $\frac{1}{1+x^2}$ in powers of $x$ and hence find a power series expansion of $\tan^{-1}x$.\\

\item Comment on the convergence of the series $\sum a_n $ where 
$
a_n=\begin{cases}
\frac{n}{2^n},& if \ n \ is \ odd \\
\frac{1}{2^n},& if \ n \ is \ even
\end{cases}
.$\\


\item Discuss the Gamma function as an improper integral with respect to its convergence and show that $\Gamma(n+1)=n!$.\\

\item Expand in Fourier series the function $f(x)=\frac{\pi^2}{12}- \frac{x^2}{4}$ in the interval $[-\pi,\pi]$ and hence show that $\frac{\pi^2}{12}= \sum\limits_{n=1}^{\infty} (-1)^{n+1}\frac{1}{n^2}$.\\

\item Comment on the convergence of the following integrals:
\begin{enumerate}[(a)]
\begin{multicols}{2}
\setlength\itemsep{2em}
\item $
\int_{0}^{1}\frac{1}{\sqrt{1-x^2}} \ dx
$
\item $
\int_{0}^{\infty}x \sin x \ dx.
$\\
\end{multicols}
\end{enumerate}

\item Consider the function 
$
f(x,y)=\begin{cases}
\frac{xy(x^2-y^2)}{x^2+y^2},& if \ (x,y)\neq(0,0) \\
0 ,& if \ (x,y)=(0,0)
\end{cases}
.$\\

\begin{itemize}
\setlength\itemsep{2em}
\item Show that $\frac{\partial^2 f}{\partial x \partial y}(x,y)=\frac{\partial^2 f}{\partial y \partial x}(x,y)$ where $(x,y)\neq(0,0)$.
\item Is $\frac{\partial^2 f}{\partial x \partial y}(0,0)=\frac{\partial^2 f}{\partial y \partial x}(0,0)$? Substantiate.\\

\end{itemize}


\item Evaluate $\int_{a}^{b} e^x \ dx$ by calculating it as the limit of Riemann sum.
\\
\item Find the $total$ area of a figure bounded by $y=x$, \ $y=2x$ and the curve $y=x^3$.
\\
\item Consider the cycloid $x=r(t-\sin t)$; \ $y=r(1-\cos t)$.\\

\begin{enumerate}[(a)]

\item Find the arc length of one arch $(0\leqslant t \leqslant 2\pi)$.
\item Find the surface area of the solid generated by rotating this arch about the $x$-axis.
\end{enumerate}


\end{enumerate}

%\end{enumerate}

\end{document}