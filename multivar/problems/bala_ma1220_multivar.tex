\documentclass[journal,12pt,twocolumn]{IEEEtran}
\usepackage{amsmath}
\usepackage{amssymb}
\usepackage{enumerate}
\usepackage[utf8]{inputenc}
\usepackage{multicol}
\usepackage{siunitx}
\usepackage{iithtlc}
\begin{document}
\providecommand{\nCr}[2]{\,^{#1}C_{#2}} % nCr
\providecommand{\nPr}[2]{\,^{#1}P_{#2}} % nPr
\providecommand{\mbf}{\mathbf}
\providecommand{\pr}[1]{\ensuremath{\Pr\left(#1\right)}}
\providecommand{\qfunc}[1]{\ensuremath{Q\left(#1\right)}}
\providecommand{\sbrak}[1]{\ensuremath{{}\left[#1\right]}}
\providecommand{\lsbrak}[1]{\ensuremath{{}\left[#1\right.}}
\providecommand{\rsbrak}[1]{\ensuremath{{}\left.#1\right]}}
\providecommand{\brak}[1]{\ensuremath{\left(#1\right)}}
\providecommand{\lbrak}[1]{\ensuremath{\left(#1\right.}}
\providecommand{\rbrak}[1]{\ensuremath{\left.#1\right)}}
\providecommand{\cbrak}[1]{\ensuremath{\left\{#1\right\}}}
\providecommand{\lcbrak}[1]{\ensuremath{\left\{#1\right.}}
\providecommand{\rcbrak}[1]{\ensuremath{\left.#1\right\}}}
\newcommand{\sgn}{\mathop{\mathrm{sgn}}}
\providecommand{\abs}[1]{\left\vert#1\right\vert}
\providecommand{\res}[1]{\Res\displaylimits_{#1}} 
\providecommand{\norm}[1]{\lVert#1\rVert}
\providecommand{\mtx}[1]{\mathbf{#1}}
\providecommand{\mean}[1]{E\left[ #1 \right]}
\providecommand{\fourier}{\overset{\mathcal{F}}{ \rightleftharpoons}}
%\providecommand{\hilbert}{\overset{\mathcal{H}}{ \rightleftharpoons}}
\providecommand{\system}{\overset{\mathcal{H}}{ \longleftrightarrow}}

\newcommand{\solution}{\noindent \textbf{Solution: }}
\providecommand{\dec}[2]{\ensuremath{\overset{#1}{\underset{#2}{\gtrless}}}}
\title{ 
\logo{
Problem Set: Multivariable Calculus
}
}
\author{J.~Balasubramaniam$^{\dagger}$ %<-this  stops a space
\thanks{$\dagger$ The author is with the Department of Mathematics, IIT Hyderabad
502285 India e-mail: jbala@iith.ac.in. }
}
\maketitle


\section{Limits and Partial Derivatives}

\begin{enumerate}


\item Find the domain and range of the following functions:

\begin{enumerate}[(i)]
\begin{multicols}{2}
{\small
\item $
f(x,y)=e^x+e^y
$

\item $
f(x,y)=\frac{x}{y}
$

\item $
f(x,y)= \cos^{-1}(x-y)
$

\item $
f(x,y)=\sqrt{\frac{(x-y)}{(x+y}}
$

\item $
f(x,y)=\frac{y}{|x|}
$

\item $
f(x,y)= \frac{x}{2y}+\frac{y}{2x}
$

\item $
f(x,y)=\frac{1}{(x^2-y^2)^\frac{3}{2}}
$

\item $
f(x,y,z)=\sqrt{-x^2-y^2-z^2}
$

\item $
f(x,y,z)=\tan^{-1} \big(\frac{x+z}{y}\big)
$

\item $
f(x,y,z)= \ln(1+x^2-y^2+z)
$
}
\end{multicols}
\end{enumerate}

\item Verify the following limits using $\epsilon $ - $\delta $ definition:

\begin{enumerate}[(i)]

\item $
\lim\limits_{(x,y) \to (1,2)} (3x+y)=5
$

\item $
\lim\limits_{(x,y) \to (0,0)} \frac{2x^2y}{x^2+y^2}=0
$

\end{enumerate}

\item Show that the following limits do not exist:

\begin{enumerate}[(i)]
\begin{multicols}{2}

\item $
\lim\limits_{(x,y) \to (0,0)} \frac{x+y}{x-y}
$

\item 
$\lim\limits_{(x,y) \to (0,0)} \frac{xy^3}{x^4+y^4}
$

\item
$\lim\limits_{(x,y) \to (1,1)} \frac{xy}{x^2+y^2}
$

\item
$\lim\limits_{(x,y,z) \to (0,0,0)} \frac{xyz}{x^3+y^3+z^3} 
$

\end{multicols}
\end{enumerate}

\item Show that the following limits exist and calculate them:

\begin{enumerate}[(i)]
\begin{multicols}{2}

\item
$\lim\limits_{(x,y) \to (0,0)} \frac{3xy}{\sqrt{x^2+y^2}}
$

\item
$\lim\limits_{(x,y) \to (0,0)} \frac{5x^2y^2}{x^4+y^2}
$

\item
$\lim\limits_{(x,y) \to (0,0)} \frac{x^3+y^3}{x^2+y^2}
$

\item
$\lim\limits_{(x,y,z) \to (0,0,0)} \frac{xyz}{x^2+y^2+z^2}
$

\end{multicols}
\end{enumerate}

\item Find the indicated values at the given points:

\begin{enumerate}

\item
$ f(x,y)= \sin (x+y); \ f_x(\pi/6,\pi/3)$

\item
$ f(x,y)= \ln (x^2+y^4); \ f_y(3,1)$

\item
$ f(x,y)= e^{\sqrt{x^2+y}}; \ f_y(0,4)$

\item
$ f(x,y)=\frac{x^2-y^2}{x^2+y^2}; \ f_y(2,-3)$

\item
$ f(x,y,z)= \sin(2xy^4z); \ f_{xz}$

\item
$ f(x,y,z)= e^{xy} \sin z; \ f_{xy}$

\item
$ f(x,y,z)= \cos(x+2y+3z); \ f_{yz}$

\item
$ f(x,y)= \ln (3x-2y); \ f_{yxy}$

\end{enumerate}
\end{enumerate}
%
\section{Continuity, Differentiability and Approximations}
\begin{enumerate}

\item Find the maximum region over which the following functions are continuous:

\begin{enumerate}[(i)]
\begin{multicols}{2}
{\small
\item $
f(x,y)=e^{xy+2}
$

\item 
{\footnotesize
$
f(x,y)=
\frac{x^3+4xy^6-7x^4}{x^3-y^3}
$
}
\item $
f(x,y)= \tan^{-1}(x-y)
$

\item $
f(x,y)= \sqrt{x-y}
$

\item $
f(x,y,z)=y \ln(xz)
$

\item 
{\footnotesize
$
f(x,y,z)=\frac{1}{\sqrt{1-x^2-y^2-z^2}}
$
}
}
\end{multicols}
\end{enumerate}


\item Find a function $g(x)$ / a number $c$ such that the following functions are continuous:

\begin{enumerate}[(a)]

\item $
 f(x,y)=\begin{cases}
\frac{x^2-y^2}{x-y},& x \neq y \\
g(x),& x = y
\end{cases}
$ .

\item $
f(x,y)=\begin{cases}
\frac{3xy}{\sqrt{x^2+y^2}},& (x,y) \neq (0,0) \\
c,& (x,y) = (0,0)
\end{cases}
$ .

\item $
f(x,y)=\begin{cases}
\frac{xy}{|x|+|y|},& (x,y) \neq (0,0) \\
c,& (x,y) = (0,0)
\end{cases}
$ .

\end{enumerate}


\item Show by definition that the following are differentiable:

\begin{enumerate}[(i)]
\begin{multicols}{2}

\item $
z=x^2+y^2
$

\item $
z=x^2y^2
$

\item $
z$ is any polynomial in $x,y
$

\end{multicols}
\end{enumerate}

\item Calculate the gradient $\nabla f$ of the following functions:
 
\begin{enumerate}[(i)]
\begin{multicols}{2}
{\small
\item $
f(x,y)={(x+y)}^2
$

\item $
f(x,y)=\frac{x-y}{x+y}
$

\item $
f(x,y)=\sqrt{x^2+y^3}
$

\item $
f(x,y)=\frac{e^{x^2} - e^{-y^2}}{3y}
$

\item 
{\footnotesize
$
f(x,y,z)=x \sin y \ln z
$
}
\item 
{\footnotesize
$
f(x,y,z)=\frac{x-z}{\sqrt{1-y^2+x^2}}
$
}
\item 
$
f(x,y,z)=x \cosh y- y \ln z
$

\item $
f(x,y,z)=(y-z)e^{\sqrt{1-y^2+x^2}}
$
}
\end{multicols}
\end{enumerate}


\item Let $f$ and $g$ be differentiable functions of two variables.

\begin{enumerate}[(a)]

\item Show that $ \nabla (f + g)$ = $ \nabla f + \nabla g$.
\item Show that $fg$ is differentiable and that $ \nabla (fg)$ = $ f \nabla g + g \nabla f $.
\item Show that $ \nabla (f) =  \bar{0}$ if and only if $f$ is a constant.
\item Show that if $ \nabla f = \nabla g$ then there is a constant $c$ such that $f(x,y) = g(x,y) = c$.
\item What is the most general function $f$ such that $ \nabla f(\bar{x}) = \bar{x}$ for every  $ \bar{x} \in \mathbb{R}^2$.

\end{enumerate}

\item Use the total differential to estimate the given numbers:

\begin{enumerate}[(i)]
\begin{multicols}{2}

\item $
\frac{3.01}{5.99}
$

\item $
\sqrt{\frac{5.02-3.96}{5.02+3.96}}
$

\item $
\sin \big(\frac{11 \pi}{24}\big) \cos \big(\frac{13 \pi}{36} \big)
$

\end{multicols}
\end{enumerate}

\item When 3 resistors $r_1,r_2,r_3$ are connected in parallel, the total resistance $R = \frac{1}{r_1}+\frac{1}{r_2}+\frac{1}{r_3}$. If $r_1 =6 \pm 0.1$,$r_2=8 \pm 0.03$ and $r_3=12 \pm 0.15$ ohms, estimate $R$ and find an approximate value for the maximum error in your estimate.

\item How much wood is contained in the sides of a rectangular box with sides of inside measurements
$1m$, $1.2m$ and $1.6m$, if the thickness of the wood making up the sides is $5cm$.

\item The volume of 10 moles of an ideal gasa was calculated to be $500cm^3$ at a temperature of $ \ang{40} C$.If the maximum error in each measurement $n$, $V$ and $T$ is $\frac{1}{2} \% $, calculate the approximate pressure of the gas ( in $nt/cm^2$), and find the approximate error in your computation.

%\end{enumerate}
\end{enumerate}


\section{Gradient and Its Applications}

\begin{enumerate}

\item Calculate the directional derivatives at the given point in the direction of  $\bar{v}$:

\begin{enumerate}[(i)]

\item $
f(x,y)=xy$  \ \ \ \ \ \ \ \ \ \ \ \ \ \ \ \ \ \ \ \ \ \ \ \  at $ (2,3) ; \bar{v}=\bar{i}+3\bar{j}
$

\item $
f(x,y)= \ln(x+3y)$ \ \ \ \ \ \ \ \ \ \ \ \ \ \ at $ (2,4) ; \bar{v}=\bar{i}+\bar{j}
$


\item $
f(x,y)= \tan^{-1} \big(\frac{y}{x}\big) $ \ \ \ \ \ \ \ \ \ \ \ \ \ \ \ at $ (2,2) ; \bar{v}=3\bar{i}-2\bar{j}
$


\item $
f(x,y,z)= x^2y^3+z\sqrt{x}  $ \ \ \ \ \ \ \ \ \  at $ (1,-2,3) ; \bar{v}=5\bar{j}+\bar{k}
$


\item $
f(x,y,z)= e^{-(x^2+y^2+z^2)} $ \ \ \ \ \ \ \ \ \ \ at $ (1,1,1) ; \bar{v}=\bar{i}+3\bar{j}-5\bar{k}
$


\item $
f(x,y,z)= \frac{1}{\sqrt{x^2+y^2+z^2}} $ \ \ \ \ \ \ \ \ \ \ \ \ at $ (-1,2,3) ; \bar{v}=\bar{i}-\bar{j}+\bar{k}
$


\end{enumerate}

\item The temperature distribution of a ball centered at the origin is given by $ T(x,y,z)=\frac{100}{x^2+y^2+z^2+1}$.

\begin{enumerate}[(a)]

\item Where is the ball hottest ?
\item Find the direction of greatest decrease of heat at the point $(3,-1,2)$ .
\item Find the direction of greatest increase in heat. Does this vector point toward the origin ?

\end{enumerate}


\item Determine the nature of the critical points of the given functions:

\begin{enumerate}[(i)]


\item $
f(x,y)=7x^2-8xy+3y^2+1
$

\item $
f(x,y)=x^2+y^3-3xy
$

\item $
f(x,y)=x^3+3xy^2+3y^2-15x+2 
$

\item $
f(x,y)=\frac{1}{y}-\frac{1}{x}-4x+y
$

\item $
f(x,y)=xy+\frac{1}{y}+\frac{8}{x}
$

\item $
f(x,y)=\frac{2}{y}+\frac{1}{x}+2x+y+1
$


\end{enumerate}

\item Find $3$ numbers whose sum is $50$ such that the product $xy^2z^3$ is a maximum.

\item What is the maximum volume of an open-top rectangular box that can be built from $\alpha$ square
meters of wood ?

\item Find the dimensions of the rectangular box of maximum volume that can be inscribed in the ellipsoid $\frac{x^2}{a^2}+\frac{y^2}{b^2}+\frac{z^2}{c^2}=1$ whose faces are parallel to the coordinate planes.

\item A company uses $2$ types of raw materials, I and II, for its product. If it uses $x$ units of I and $y$ units of II it can produce $U$ units of the finished item where $U(x,y)=8xy+32x+40y-4x^2-6y^2$.
Each unit of I costs Rs. $10$ and each of unit of II costs Rs. $4$. Each unit of the product can be sold for Rs. $40$. How can the company maximize its profits?

\item Solve the following using Lagrange multipliers:

\begin{enumerate}[(a)]

\item Minimum distance from the point $(3,0,1)$  to the plane $2x-y+4z=5$.

\item Find the maximum and minimum values of $xyz$ if $(x,y,z)$ is on the ellipsoid $\frac{x^2}{a^2}+\frac{y^2}{b^2}+\frac{z^2}{c^2}=1$.

\item Maximize the function $x^3+y^3+z^3$ for $(x,y,z)$ on the planes $x+y+z=2$ and $x+y-z=3$.

\item Show that among all triangles having the same perimeter, the equilateral triangle has the
greatest area.

\item The plane $x+y+z=1$ cuts the cylinder $x^2+y^2=1$ in an ellipse. Find the points on the ellipse that are closest and farthest from the origin.

\end{enumerate}

\item Find $\frac{\partial z}{\partial x} $ and $\frac{\partial z}{\partial y}$ at the indicated points:

\begin{enumerate}[(i)]

\item $z^3-xy+yz+y^3-2=0 $ at $(1,1,1)$

\item $\frac{1}{x}+\frac{1}{y}+\frac{1}{z}-1=0 $  at $(2,3,6)$

\item $ xe^y+ye^z+2 \ln x-2-3 \ln 2=0 $  at $(1, \ln 2, \ln 3)$.

\end{enumerate}


\end{enumerate}

\section{Exercises}

\begin{enumerate}
\item Give examples of the following, with a brief justification. 

\begin{enumerate}[(a)]
 
\item A function $f(x,y)$ and a point $(x_0,y_0)$ such that $(x_0,y_0)$ is a saddle point.
\item A function $f(x,y)$ such that $\Delta f = 9x^2y^2 \hat{i}+6 x^3y \hat{j}.$
\item A function $f(x,y)$ such that at the point $(3,1)$ the direction of steepest increase of $f$ is along the vector $\bar{u}=\hat{i}+6\hat{j}$.

\end{enumerate}

 
\item Use total differential to estimate $(1.95)^4 (3.04)^3 (0.97)^5$.


\item Show by definition that $f(x,y) = xy^2$ is differentiable. 

\item Determine the nature of the critical points of $ f(x,y) = 9x^2+4y^2-12xy+7 $.

\item A firm has Rs. $250,000$ to spend on labour and raw materials. The output of the firm is $\alpha xy$
where $\alpha$ is a constant and $x$,$y$ are the quantity of labour and raw materials consumed. If the unit price of hiring labour is Rs. $5000$ and the unit price of raw materials is $2500$, find the ratio of $x$ to $y$ that maximizes output.

\item Find the point at which $w = xyz$ attains its maximum w.r.to the constraints $x + y + z = 30$ and
$x + y = z$.

\item Show that every line normal to the surface of a sphere passes through its center.

\item Give examples of functions with the following properties: 

\begin{enumerate}[(a)]
 
\item A function $f(x,y)$ whose domain is $\lbrace (x, y) \in \mathbb{R}^2 | x \neq 0 \rbrace$ and whose range is $(\frac{-\pi}{2},\frac{\pi}{2})$.


\item A function $f(x,y)$ such that $\frac{df}{dx}$ exists at $(0,0)$ but $f$ is not continuous at $(0,0)$.

\item  A function $f(x,y)$ such that $f_{xy}(0,0) \neq f_{yx}(0,0)$.

\item A function $f(x,y)$ such that $f(0,0)$ is defined but $\nabla f(0,0)$ is not.


\end{enumerate}

\item Find the domain, range and level curves of $f(x,y) = \frac{x^2-y^2}{x^2+y^2}$.


\item Show that $\lim\limits_{(x,y) \to (0,0)} \frac{y^2-2x}{y^2+2x}$ does not exist.

\item Find the unit vector along which $4x^2+9y^2$ increases most rapidly at the point $(2,1)$.

\item Find a number such that the following function is continuous at the origin:

$ 
f_(x,y)=\begin{cases}
\frac{-2xy}{\sqrt{x^2+y^2}},& (x,y) \neq (0,0) \\
c,& (x,y) = (0,0)
\end{cases}
$ .


\item Examine the function $y = 2 \sin x + \cos 2x$ for maxima and minima.

\item Let $z = f(x,y)$ with $x = r \cos \theta$ and $y = r \sin \theta$. Also writing $z = g(r,\theta) = f(r \cos \theta, r \sin \theta)$, show that

$\Big(\frac{\delta g}{\delta r}\Big)^2 + \frac{1}{r^2} \Big(\frac{\delta g}{\delta \theta}\Big)^2 = \Big(\frac{\delta f}{\delta x}\Big)^2 + \Big(\frac{\delta f}{\delta y}\Big)^2$.

\item Determine the nature of the critical points of $f(x,y) = 2x^3-24xy+16y^3$.

\item How much wood is contained in the sides of a rectangular box with sides of inside measurements
$1.5m, 1.3m \ and \ 2m$, if the thickness of the wood making up the sides is $3cm$?

\item A silo is in the shape of a cylinder topped with a cone (much like a circus tent). If the radius of
each is $6m$, and the total surface area is $200 m^2$ (excluding the base), what are the heights of the
cylinder and the cone that maximize the volume enclosed by the silo?


\end{enumerate}
\end{document}

